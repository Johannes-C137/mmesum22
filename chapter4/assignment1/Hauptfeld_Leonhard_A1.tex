\title{Assignment 1 - Biological Databases}
\author{
        Leonhard Hauptfeld\\
        MME2 - Bioinformatics
}
\date{\today}

\documentclass[12pt]{article}
\usepackage[english]{babel}

% Set \parskip to put 10pt between paragraphs
\setlength{\parskip}{10pt}

% Set the value of \parindent to 0pt
\setlength{\parindent}{0pt}

\usepackage[a4paper,
            bindingoffset=0.2in,
            left=1cm,
            right=1cm,
            top=1cm,
            bottom=1cm,
            footskip=.25in]{geometry}

\usepackage{multirow}

\usepackage{newunicodechar} % To write the definition on the next line
\newunicodechar{α}{\ensuremath{\alpha}}
\newunicodechar{β}{\ensuremath{\beta}}
\newunicodechar{γ}{\ensuremath{\gamma}}

\usepackage{titling}
\setlength{\droptitle}{-2cm}

\usepackage{outlines}
% Good underlining with \ul instead of shitty LaTeX \underline
\usepackage{soul}

% References
\usepackage[style=ieee]{biblatex}
\addbibresource{\jobname.bib}

\begin{document}
\maketitle

\section{PubMed}
The chosen topic was \textbf{multiple sclerosis (MS)}.

An initial search for the topic yielded 100,616 results.
After filtering for a publication age of 5-10 years (2012-2017) and 
abstract availability, the result count reduced to 23,755. Further
narrowing down the results to only include articles of the type "Randomized Control Trials"
the search yielded 807 results. The search was further optimized by applying the keywords "genetic", "gene" or "biomarker" to only include
abstracts that mentioned genetic relations to MS.

The following 5 abstracts were picked from the results to be further analyzed for goals, methods
and relevant genes mentioned.
\begin{outline}
    \1 \ul{\textbf{The role of KIR2DS1 in multiple sclerosis--KIR in Portuguese MS patients}}\cite{killercells}
      \2 \textbf{Goals:} Investigating the influence of an activated KIR2DS1 gene regarding a person's resistance \& susceptibility to MS
      \2 \textbf{Methods:} Observing activation of KIR2DS1 in 447 MS Portuguese independent from the presence of HLA-DRB1*15 allele.
      \2 \textbf{Genes:} HLA-DRB1*15, KIR2DS1
    \1 \ul{\textbf{Impact of vitamin A supplementation on RAR gene expression in multiple sclerosis patients}}\cite{vitaminA}
      \2 \textbf{Goals:} Exploring the impact of retinyl palmitate supplementation on RAR subtype gene expression in peripheral blood mononuclear cells (PBMCs) with multiple sclerosis (MS) patients
      \2 \textbf{Methods:} Double-blind randomized clinical trial, placebo and intervention group were given Vitamin-D and interferon beta-1a, intervention group would receive
      additional retinyl palmitate whereas the placebo group would not. Peripheral blood mononuclear cells (PBMC) would be extracted and expression of the RAR-genes tested by PCR.
      \2 \textbf{Genes:} RAR-α, RAR-γ
\newpage
    \1 \ul{\textbf{The Effect of Vitamin A Supplementation on FoxP3 and TGF-}}β \ul{\textbf{Gene Expression in Avonex-Treated Multiple Sclerosis Patients}}\cite{vitaminA2}
      \2 \textbf{Goals:} Evaluating the impact of vitamin A supplementation on T cell balance (expression of FoxP3 and TGF-β)
      \2 \textbf{Methods:} Clinical trial with MS patients, an intervention group receiving vitamin A supplements and a placebo group taking placebos.
      Measurement of gene expression in PBMC through PCR.
      \2 \textbf{Genes:} FoxP3, TGF-β
    \1 \ul{\textbf{Vitamin D, HLA-DRB1 and Epstein-Barr virus antibody levels in a prospective cohort of multiple sclerosis patients}}\cite{epbarr}
      \2 \textbf{Goals:} To study the association between serum levels of anti Epstein-Barr virus nuclear antigen 1 (EBNA-1) antibody and 25-hydroxyvitamin D (25(OH)D) in a prospective cohort of patients with relapsing-remitting multiple sclerosis
      \2 \textbf{Methods:} 90 patients with relapsing-remitting multiple sclerosis, all participants in a randomized clinical trial; 
      repeated, paired measurements of serum 25(OH)D and serum EBNA-1 immunoglobulin G (IgG) levels;
      association between serum EBNA-1 IgG and 25(OH)D levels and HLA-DRB1*15 positive status analysed 
      \2 \textbf{Genes:} HLA-DR1B
    \1 \ul{\textbf{In vivo maintenance of human regulatory T cells during CD25 blockade}}\cite{tcells}
      \2 \textbf{Goals:} Investigating the effect of mediated CD25 blockade on Treg homeostasis in patients with relapsing-remitting multiple sclerosis
      \2 \textbf{Methods:} Monitoring the effect over a 52-week randomized controlled trial
      \2 \textbf{Genes:} IFN-g, FOXP3, STAT5, IL2RA
\end{outline}

\section{GenBank}
\paragraph{Accession Number} The accession number is a unique identifier for a DNA sequence.
For GenBank, they are usually comprised of a prefix (letters) and a number.\cite{accno}

\paragraph{Orthologs}"Orthologs are genes in different species
that evolved from a common ancestral gene by speciation, and, in general, orthologs retain the same function during the course of evolution. Identification of orthologs is a critical process for reliable prediction of gene function in newly sequenced genomes." \cite{ortho}

What follows on the next page is a more detailed analysis of the genes identified in the 5 studies from chapter 1.

\begin{table}[h]
  \begin{tabular}{|p{2.6cm}|p{1cm}|p{3cm}|p{1.8cm}|p{3.5cm}|p{3.8cm}|}
  \hline
  \textbf{Gene}     & \textbf{Chr.} & \textbf{Position} & \textbf{Species} & \textbf{Function(s)} & \textbf{Protein Family} \\ \hline
  \textbf{HLA-DR1B} & 6  & NC\_000006.12 (32578775.. 32589848, complement) & Homo sapiens & Immune System, presenting peptides derived from extracellular proteins & Human leukocyte antigen complex (HLA), distinguishing body proteins from invading proteins \\ \hline
  \textbf{KIR2DS1}  & 19 & NC\_037345.1 (62754403.. 62766895, complement)  & Homo sapiens, Bos taurus & Immune System, regulation of the immune response & Killer cell immunoglobulin-like receptors (KIR), transmembrane glycoproteins expressed by natural killer cells and subsets of T cells \\ \hline
  \textbf{RAR-α}    & 17 & NC\_000017.11 (40309180.. 40357643)             & Homo sapiens, Mus musculus, etc. & Transcription, regulates transcription & Retinoic acid receptors (RARs), active/inhibit transcription dependent on binding \\ \cline{1-3}
  \textbf{RAR-γ}    & 12 & NC\_000012.12 (53210569.. 53232231, complement) & & & \\ \hline
  \textbf{FoxP3}    & X  & NC\_000023.11 (49250436.. 49264932, complement) & Homo sapiens, Mus musculus, etc. & Transcription, regulates transcription, defects linked to immunodeficiency  & Human Forkhead-box (FOX), linked to transcription \\ \hline
  \textbf{STAT5}    & 17 & NC\_000017.11 (42287547.. 42311943)             & Homo sapiens, Mus musculus, etc. & Transcription (Growth, Lactation, Defense) & Signal Transducers and Activators of Transcription (STAT), transcription factors \\ \hline
  \textbf{TGF-β}    & 19 & NC\_000019.10 (41330323.. 41353922, complement) & Homo sapiens, Mus musculus, etc. & Transcription, Ligand encoding & Transforming Growth Factors (TGF), encoding of ligands to do with various SMAD transcription factors \\ \hline
  \textbf{IFN-g}    & 12 & NC\_000012.12 (68154768.. 68159740, complement) & Homo sapiens, Mus musculus, etc. & Immune System, encodes a protein that triggers immune response & Interferon (IFN) protein encoders, immune response signalling \\ \hline
  \textbf{IL2RA}    & 10 & NC\_000010.11 (6010689.. 6062367, complement)    & Homo sapiens, Mus musculus, etc. & Immune System, encodes a protein that triggers immune response & Interleukin (IL) protein encoders, immune response signalling  \\ \hline
  \end{tabular}
\end{table}
\clearpage
\printbibliography

\end{document}
